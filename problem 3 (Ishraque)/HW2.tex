\documentclass{article}\usepackage[]{graphicx}\usepackage[]{color}
% maxwidth is the original width if it is less than linewidth
% otherwise use linewidth (to make sure the graphics do not exceed the margin)
\makeatletter
\def\maxwidth{ %
  \ifdim\Gin@nat@width>\linewidth
    \linewidth
  \else
    \Gin@nat@width
  \fi
}
\makeatother

\definecolor{fgcolor}{rgb}{0.345, 0.345, 0.345}
\newcommand{\hlnum}[1]{\textcolor[rgb]{0.686,0.059,0.569}{#1}}%
\newcommand{\hlstr}[1]{\textcolor[rgb]{0.192,0.494,0.8}{#1}}%
\newcommand{\hlcom}[1]{\textcolor[rgb]{0.678,0.584,0.686}{\textit{#1}}}%
\newcommand{\hlopt}[1]{\textcolor[rgb]{0,0,0}{#1}}%
\newcommand{\hlstd}[1]{\textcolor[rgb]{0.345,0.345,0.345}{#1}}%
\newcommand{\hlkwa}[1]{\textcolor[rgb]{0.161,0.373,0.58}{\textbf{#1}}}%
\newcommand{\hlkwb}[1]{\textcolor[rgb]{0.69,0.353,0.396}{#1}}%
\newcommand{\hlkwc}[1]{\textcolor[rgb]{0.333,0.667,0.333}{#1}}%
\newcommand{\hlkwd}[1]{\textcolor[rgb]{0.737,0.353,0.396}{\textbf{#1}}}%
\let\hlipl\hlkwb

\usepackage{framed}
\makeatletter
\newenvironment{kframe}{%
 \def\at@end@of@kframe{}%
 \ifinner\ifhmode%
  \def\at@end@of@kframe{\end{minipage}}%
  \begin{minipage}{\columnwidth}%
 \fi\fi%
 \def\FrameCommand##1{\hskip\@totalleftmargin \hskip-\fboxsep
 \colorbox{shadecolor}{##1}\hskip-\fboxsep
     % There is no \\@totalrightmargin, so:
     \hskip-\linewidth \hskip-\@totalleftmargin \hskip\columnwidth}%
 \MakeFramed {\advance\hsize-\width
   \@totalleftmargin\z@ \linewidth\hsize
   \@setminipage}}%
 {\par\unskip\endMakeFramed%
 \at@end@of@kframe}
\makeatother

\definecolor{shadecolor}{rgb}{.97, .97, .97}
\definecolor{messagecolor}{rgb}{0, 0, 0}
\definecolor{warningcolor}{rgb}{1, 0, 1}
\definecolor{errorcolor}{rgb}{1, 0, 0}
\newenvironment{knitrout}{}{} % an empty environment to be redefined in TeX

\usepackage{alltt}
\usepackage{amsmath} %This allows me to use the align functionality.
                     %If you find yourself trying to replicate
                     %something you found online, ensure you're
                     %loading the necessary packages!
\usepackage{amsfonts}%Math font
\usepackage{graphicx}%For including graphics
\usepackage{hyperref}%For Hyperlinks
\hypersetup{colorlinks = true,citecolor=black}
\usepackage{natbib}        %For the bibliography
\bibliographystyle{apalike}%For the bibliography
\usepackage[margin=0.5in]{geometry}
\usepackage{float}
\IfFileExists{upquote.sty}{\usepackage{upquote}}{}
\begin{document}
\noindent \textbf{MA 354: Data Analysis -- Fall 2021 -- Due 10/8 at 5p}\\%\\ gives you a new line
\noindent \textbf{Homework 2:}\vspace{1em}\\
\emph{Complete the following opportunities to use what we've talked about in class. These questions will be graded for correctness, communication and succinctness. Ensure you show your work and explain your logic in a legible and refined submission.}\\\vspace{1em}
%Comments -- anything after % is not put into the PDF

The starting jobs will be applied in alphabetical order (last name) for question two.
\begin{enumerate}
  \item \textbf{Solver:} provide a solution, if possible, and reasoning for the solution. \textbf{Due to group 10/5 or earlier.}
  \item \textbf{Code Checker:} provides a first check of the solver's worked solutions and ensures they are correct with a solid interpretation. 
  \item \textbf{Checker} checks the solution for completeness, proposes and implements changes if agreed upon by the group. Provides a short paragraph summarizing the discussion of proposals and their reason for acceptance or non-acceptance.
  \item \textbf{Double Checker} checks the solution for completeness, communication and polish. The Double Checker ensures that the solution is correct and highly polished for submission.
\end{enumerate}

\noindent For subsequent questions student roles will move down one position. The rolls change as follows.
\begin{enumerate}
  \item \textbf{Solver} $\Longrightarrow$ \textbf{Code Checker}
  \item \textbf{Code Checker} $\Longrightarrow$ \textbf{Checker}
  \item \textbf{Checker} $\Longrightarrow$ \textbf{Double Checker}
  \item \textbf{Double Checker} $\Longrightarrow$ \textbf{Solver}
\end{enumerate}
While students have assigned jobs for each question I encourage students to help 
each other complete the homework in collaboration.
\newpage
\begin{enumerate}
%%%%%%%%%%%%%%%%%%%%%%%%%%%%%%%%%%%%%%%%%%%%%%%%%%%%%%%%%%%%%%%%%%%%%%%%%%%%%%%%%%%%%%%%%%%
%%%%%%%%%%%%%%%%%%%%%%%%%%%%%%%%%%%%%%%%%%%%%%%%%%%%%%%%%%%%%%%%%%%%%%%%%%%%%%%%%%%%%%%%%%%
%%%%%%%%%  Question 1
%%%%%%%%%%%%%%%%%%%%%%%%%%%%%%%%%%%%%%%%%%%%%%%%%%%%%%%%%%%%%%%%%%%%%%%%%%%%%%%%%%%%%%%%%%%
%%%%%%%%%%%%%%%%%%%%%%%%%%%%%%%%%%%%%%%%%%%%%%%%%%%%%%%%%%%%%%%%%%%%%%%%%%%%%%%%%%%%%%%%%%%
  \item\label{Q1} Select a continuous distribution (Not the uniform or exponential). 
  It does not have to be one that we cover in the notes! To explore the PDF of your 
  distribution, specify two sets of parameter(s) for your distribution.
  \begin{enumerate}
  %%%%%%%%%%%%%%%%%%%%%%%%%%%%%%%%%%%%%%%%%%%%%%%%%%%%%%%%%%%%%%%%%%%%%%%%%%%%%%%%%%%%%%%%%%%
  %%%%%%%%%  Part (a)
  %%%%%%%%%%%%%%%%%%%%%%%%%%%%%%%%%%%%%%%%%%%%%%%%%%%%%%%%%%%%%%%%%%%%%%%%%%%%%%%%%%%%%%%%%%%
  \item \textbf{History} Discuss what types of random variables are modeled with 
  your distribution. Be sure to include a discussion about the support and ensure 
  to provide the density function, and CDF. This requires some internet research 
  -- what's the history of the distribution, why was it created and named? What 
  are some exciting applications of this distribution?
  
  Cite all of your sources in LaTeX by adding a BibTeX citation to the .bib file. 
  To help, I've cited R \citep{R21} in parentheses here. \cite{R21} provides helpful 
  tools for the rest of the questions below. BibTeX citations are available through 
  Google Scholar by clicking the cite button below the article of  interest and 
  selecting the BibTeX option.
  %%%%%%%%%%%%%%%%%%%%%%%%%%%%%%%%%%%%%%%%%%%%%%%%%%%%%%%%%%%%%%%%%%%%%%%%%%%%%%%%%%%%%%%%%%%
  %%%%%%%%%  Part (b)
  %%%%%%%%%%%%%%%%%%%%%%%%%%%%%%%%%%%%%%%%%%%%%%%%%%%%%%%%%%%%%%%%%%%%%%%%%%%%%%%%%%%%%%%%%%%
	\item Show that you have a valid PDF. You will find the \texttt{integrate()} 
	function in \texttt{R} helpful.
	%%%%%%%%%%%%%%%%%%%%%%%%%%%%%%%%%%%%%%%%%%%%%%%%%%%%%%%%%%%%%%%%%%%%%%%%%%%%%%%%%%%%%%%%%%%
  %%%%%%%%%  Part (c)
  %%%%%%%%%%%%%%%%%%%%%%%%%%%%%%%%%%%%%%%%%%%%%%%%%%%%%%%%%%%%%%%%%%%%%%%%%%%%%%%%%%%%%%%%%%%
	\item Find the median for your two sets of parameter(s). Conduct some research 
	to find the median based on our PDF to confirm that your numerical approach is 
	correct. 
	%%%%%%%%%%%%%%%%%%%%%%%%%%%%%%%%%%%%%%%%%%%%%%%%%%%%%%%%%%%%%%%%%%%%%%%%%%%%%%%%%%%%%%%%%%%
  %%%%%%%%%  Part (d)
  %%%%%%%%%%%%%%%%%%%%%%%%%%%%%%%%%%%%%%%%%%%%%%%%%%%%%%%%%%%%%%%%%%%%%%%%%%%%%%%%%%%%%%%%%%%
	\item \label{q1PDF} Graph the PDF for several values of the parameter(s) 
	including the two sets you specified. What does changing the parameter(s) do 
	to the shape of the PDF?
	%%%%%%%%%%%%%%%%%%%%%%%%%%%%%%%%%%%%%%%%%%%%%%%%%%%%%%%%%%%%%%%%%%%%%%%%%%%%%%%%%%%%%%%%%%%
  %%%%%%%%%  Part (e)
  %%%%%%%%%%%%%%%%%%%%%%%%%%%%%%%%%%%%%%%%%%%%%%%%%%%%%%%%%%%%%%%%%%%%%%%%%%%%%%%%%%%%%%%%%%%
	 \item Graph the CDF for the same values of the parameter(s) as you did in 
	 Question \ref{q1PDF}. What does changing the parameter(s) do to the shape of 
	 the CDF? Comment on the aspects of the CDFs that show that the CDF is valid.
	%%%%%%%%%%%%%%%%%%%%%%%%%%%%%%%%%%%%%%%%%%%%%%%%%%%%%%%%%%%%%%%%%%%%%%%%%%%%%%%%%%%%%%%%%%%
  %%%%%%%%%  Part (f)
  %%%%%%%%%%%%%%%%%%%%%%%%%%%%%%%%%%%%%%%%%%%%%%%%%%%%%%%%%%%%%%%%%%%%%%%%%%%%%%%%%%%%%%%%%%%
  \item Generate a random sample of size $n=10, 25, 100$, and $1000$ for your 
  two sets of parameter(s). In a $4 \times 2$ grid, plot a histogram of each set
  of data and superimpose the true density function at the specified parameter 
  values. Interpret the results.
	\end{enumerate}
%%%%%%%%%%%%%%%%%%%%%%%%%%%%%%%%%%%%%%%%%%%%%%%%%%%%%%%%%%%%%%%%%%%%%%%%%%%%%%%%%%%%%%%%%%%
%%%%%%%%%%%%%%%%%%%%%%%%%%%%%%%%%%%%%%%%%%%%%%%%%%%%%%%%%%%%%%%%%%%%%%%%%%%%%%%%%%%%%%%%%%%
%%%%%%%%%  Question 2
%%%%%%%%%%%%%%%%%%%%%%%%%%%%%%%%%%%%%%%%%%%%%%%%%%%%%%%%%%%%%%%%%%%%%%%%%%%%%%%%%%%%%%%%%%%
%%%%%%%%%%%%%%%%%%%%%%%%%%%%%%%%%%%%%%%%%%%%%%%%%%%%%%%%%%%%%%%%%%%%%%%%%%%%%%%%%%%%%%%%%%%
\item Continue with the continuous distribution you selected for Question \ref{Q1}.
\begin{enumerate}
  %%%%%%%%%%%%%%%%%%%%%%%%%%%%%%%%%%%%%%%%%%%%%%%%%%%%%%%%%%%%%%%%%%%%%%%%%%%%%%%%%%%%%%%%%%%
  %%%%%%%%%  Part (a)
  %%%%%%%%%%%%%%%%%%%%%%%%%%%%%%%%%%%%%%%%%%%%%%%%%%%%%%%%%%%%%%%%%%%%%%%%%%%%%%%%%%%%%%%%%%%
  \item Provide the mean, standard deviation, skewness, and kurtosis of the PDF.
  Ensure to interpret each.
  %%%%%%%%%%%%%%%%%%%%%%%%%%%%%%%%%%%%%%%%%%%%%%%%%%%%%%%%%%%%%%%%%%%%%%%%%%%%%%%%%%%%%%%%%%%
  %%%%%%%%%  Part (b)
  %%%%%%%%%%%%%%%%%%%%%%%%%%%%%%%%%%%%%%%%%%%%%%%%%%%%%%%%%%%%%%%%%%%%%%%%%%%%%%%%%%%%%%%%%%%
  \item Generate a random sample of size $n=10, 25, 100$, and $1000$ for your 
  two sets of parameter(s). Calculate the sample mean, standard deviation, 
  skewness, and kurtosis. Interpret the results.
  %%%%%%%%%%%%%%%%%%%%%%%%%%%%%%%%%%%%%%%%%%%%%%%%%%%%%%%%%%%%%%%%%%%%%%%%%%%%%%%%%%%%%%%%%%%
  %%%%%%%%%  Part (c)
  %%%%%%%%%%%%%%%%%%%%%%%%%%%%%%%%%%%%%%%%%%%%%%%%%%%%%%%%%%%%%%%%%%%%%%%%%%%%%%%%%%%%%%%%%%%
  \item Generate a random sample of size $n=10$ for your two sets of parameter(s).
  Calculate the method of moments estimator(s) and maximum likelihood estimator(s).
  In a $1 \times 2$ grid, plot a histogram of each set of data with (1) the method 
  of moments estimated distribution, (2) the maximum likelihood estimated 
  distribution, and superimpose the true distribution in both.
  %%%%%%%%%%%%%%%%%%%%%%%%%%%%%%%%%%%%%%%%%%%%%%%%%%%%%%%%%%%%%%%%%%%%%%%%%%%%%%%%%%%%%%%%%%%
  %%%%%%%%%  Part (d)
  %%%%%%%%%%%%%%%%%%%%%%%%%%%%%%%%%%%%%%%%%%%%%%%%%%%%%%%%%%%%%%%%%%%%%%%%%%%%%%%%%%%%%%%%%%%
  \item Generate a random sample of size $n=25$ for your two sets of parameter(s).
  Calculate the method of moments estimator(s) and maximum likelihood estimator(s). 
  In a $1 \times 2$ grid, plot a histogram of each set of data with (1) the method 
  of moments estimated distribution, (2) the maximum likelihood estimated distribution, 
  and superimpose the true distribution in both.
  %%%%%%%%%%%%%%%%%%%%%%%%%%%%%%%%%%%%%%%%%%%%%%%%%%%%%%%%%%%%%%%%%%%%%%%%%%%%%%%%%%%%%%%%%%%
  %%%%%%%%%  Part (e)
  %%%%%%%%%%%%%%%%%%%%%%%%%%%%%%%%%%%%%%%%%%%%%%%%%%%%%%%%%%%%%%%%%%%%%%%%%%%%%%%%%%%%%%%%%%%
  \item Generate a random sample of size $n=100$ for your two sets of parameter(s). 
  Calculate the method of moments estimator(s) and maximum likelihood estimator(s).
  In a $1 \times 2$ grid, plot a histogram of each set of data with (1) the method 
  of moments estimated distribution, (2) the maximum likelihood estimated distribution,
  and superimpose the true distribution in both.
  %%%%%%%%%%%%%%%%%%%%%%%%%%%%%%%%%%%%%%%%%%%%%%%%%%%%%%%%%%%%%%%%%%%%%%%%%%%%%%%%%%%%%%%%%%%
  %%%%%%%%%  Part (f)
  %%%%%%%%%%%%%%%%%%%%%%%%%%%%%%%%%%%%%%%%%%%%%%%%%%%%%%%%%%%%%%%%%%%%%%%%%%%%%%%%%%%%%%%%%%%
  \item Generate a random sample of size $n=100$ for your two sets of parameter(s). 
  Calculate the method of moments estimator(s) and maximum likelihood estimator(s). 
  In a $1 \times 2$ grid, plot a histogram of each set of data with (1) the method 
  of moments estimated distribution, (2) the maximum likelihood estimated distribution, 
  and superimpose the true distribution in both.
  %%%%%%%%%%%%%%%%%%%%%%%%%%%%%%%%%%%%%%%%%%%%%%%%%%%%%%%%%%%%%%%%%%%%%%%%%%%%%%%%%%%%%%%%%%%
  %%%%%%%%%  Part (g)
  %%%%%%%%%%%%%%%%%%%%%%%%%%%%%%%%%%%%%%%%%%%%%%%%%%%%%%%%%%%%%%%%%%%%%%%%%%%%%%%%%%%%%%%%%%%
  \item Comment on the results of parts (c)-(f). 
\end{enumerate}
\newpage
%%%%%%%%%%%%%%%%%%%%%%%%%%%%%%%%%%%%%%%%%%%%%%%%%%%%%%%%%%%%%%%%%%%%%%%%%%%%%%%%%%%%%%%%%%%
%%%%%%%%%%%%%%%%%%%%%%%%%%%%%%%%%%%%%%%%%%%%%%%%%%%%%%%%%%%%%%%%%%%%%%%%%%%%%%%%%%%%%%%%%%%
%%%%%%%%%  Question 3
%%%%%%%%%%%%%%%%%%%%%%%%%%%%%%%%%%%%%%%%%%%%%%%%%%%%%%%%%%%%%%%%%%%%%%%%%%%%%%%%%%%%%%%%%%%
%%%%%%%%%%%%%%%%%%%%%%%%%%%%%%%%%%%%%%%%%%%%%%%%%%%%%%%%%%%%%%%%%%%%%%%%%%%%%%%%%%%%%%%%%%%
  \item\label{Q3} Select a discrete distribution (not the Poisson). It does not 
  have to be one that we cover in the notes! To explore the PMF of your distribution, 
  specify two sets of parameter(s) for your distribution.
  \begin{enumerate}
  %%%%%%%%%%%%%%%%%%%%%%%%%%%%%%%%%%%%%%%%%%%%%%%%%%%%%%%%%%%%%%%%%%%%%%%%%%%%%%%%%%%%%%%%%%%
  %%%%%%%%%  Part (a)
  %%%%%%%%%%%%%%%%%%%%%%%%%%%%%%%%%%%%%%%%%%%%%%%%%%%%%%%%%%%%%%%%%%%%%%%%%%%%%%%%%%%%%%%%%%%
  \item \textbf{History} Discuss what types of random variables are modeled with 
  your distribution. Be sure to include a discussion about the support and ensure
  to provide the mass function, and CDF. This requires some internet research -- 
  what's the history of the distribution, why was it created and named? What are
  some exciting applications of this distribution? Cite all of your sources.\\
  
\textbf{Solution:} The discrete distribution we have chosen is the Bernoulli distribution. The Bernoulli distribution  is a discrete probability distribution for a Bernoulli trial - a probabilistic experiment that can have one of two outcomes, success $\mathrm{(x = 1)}$ and failure $\mathrm{(x = 0)}$, and in which the probability fo success is $p$. Often $p$  is called the Bernoulli probability parameter. In Bernoulli distribution, the random variable $X$ can have only one of two values: 0 or 1. This means that the support of our discrete random variable is the set $\mathrm{{0, 1}}$. This distribution can be summarized as follows:

\begin{align*}
  p               &\in (0,1)                                                                               &\text{\textbf{[Parameter]}}\\
  \mathcal{X}     & = \{x: x \in \{0,1\}\}                                                                   &\text{\textbf{[Support]}}\\
  f_{X}(x \mid p) & = p^{x} (1-p)^{1-x}I(x \in \{0,1\})                                                      &\text{\textbf{[PMF]}}\\
  F_{X}(x \mid p) & = P(X \leq \left\lfloor x \right\rfloor)\\
                  & = [(1-p)I(\left\lfloor x \right\rfloor = 0)] + I(\left\lfloor x \right\rfloor \geq 1)  &\text{\textbf{[CDF]}}\\
\end{align*}

Simply, the Bernoulli distribution can be thought of as a model for the set of possible outcomes of any single experiment that asks a yes-no question. This distribution is named after the 17th century Swiss mathematician Jacob Bernoulli, because he was the one who explicitly defined the concept of Bernoulli trial (in his book \emph{Ars Conjectandi}) described above. 

Since the Bernoulli distribution is not cataloged by R, we have to introduce it into our calculations with the  following functions:

\begin{knitrout}
\definecolor{shadecolor}{rgb}{0.969, 0.969, 0.969}\color{fgcolor}\begin{kframe}
\begin{alltt}
\hlcom{# Bernoulli PMF}
\hlstd{dbern}\hlkwb{<-}\hlkwa{function}\hlstd{(}\hlkwc{x}\hlstd{,}\hlkwc{prob}\hlstd{)\{}
  \hlkwa{if}\hlstd{(prob}\hlopt{<}\hlnum{0} \hlopt{|} \hlstd{prob}\hlopt{>}\hlnum{1}\hlstd{)\{}
    \hlstd{errormsg} \hlkwb{<-} \hlstr{"This function is only valid for success probabilities between 0 and 1."}
    \hlkwd{stop}\hlstd{(errormsg)}
  \hlstd{\}}
  \hlstd{indicator} \hlkwb{<-} \hlkwd{rep}\hlstd{(}\hlnum{0}\hlstd{,} \hlkwd{length}\hlstd{(x))}
  \hlstd{indicator[x}\hlopt{==}\hlnum{0}\hlstd{]} \hlkwb{<-} \hlnum{1} \hlcom{# indicator should be one if x=0}
  \hlstd{indicator[x}\hlopt{==}\hlnum{1}\hlstd{]} \hlkwb{<-} \hlnum{1} \hlcom{# indicator should be one if x=1}
  \hlstd{fx} \hlkwb{<-} \hlstd{(prob}\hlopt{^}\hlstd{x} \hlopt{*} \hlstd{(}\hlnum{1}\hlopt{-}\hlstd{prob)}\hlopt{^}\hlstd{(}\hlnum{1}\hlopt{-}\hlstd{x))} \hlopt{*} \hlstd{indicator} \hlcom{# PMF formula}
  \hlkwd{return}\hlstd{(fx)}
\hlstd{\}}
\hlcom{# Bernoulli CDF}
\hlstd{pbern}\hlkwb{<-}\hlkwa{function}\hlstd{(}\hlkwc{q}\hlstd{,} \hlkwc{prob}\hlstd{)\{}
  \hlkwa{if}\hlstd{(prob}\hlopt{<}\hlnum{0} \hlopt{|} \hlstd{prob}\hlopt{>}\hlnum{1}\hlstd{)\{}
    \hlstd{errormsg}\hlkwb{<-}\hlstr{"This function is only valid for success probabilities between 0 and 1."}
    \hlkwd{stop}\hlstd{(errormsg)}
  \hlstd{\}}
  \hlstd{indicator1} \hlkwb{<-} \hlkwd{rep}\hlstd{(}\hlnum{1}\hlstd{,} \hlkwd{length}\hlstd{(q))}
  \hlstd{indicator1[q} \hlopt{!=} \hlnum{0}\hlstd{]} \hlkwb{<-} \hlnum{0} \hlcom{#indicator should be zero if x!=0}
  \hlstd{indicator2} \hlkwb{<-} \hlkwd{rep}\hlstd{(}\hlnum{1}\hlstd{,} \hlkwd{length}\hlstd{(q))}
  \hlstd{indicator2[q} \hlopt{<} \hlnum{1}\hlstd{]} \hlkwb{<-} \hlnum{0} \hlcom{#indicator should be zero if x<1}
  \hlstd{Fx} \hlkwb{<-} \hlstd{(}\hlnum{1}\hlopt{-}\hlstd{prob)} \hlopt{*} \hlstd{indicator1} \hlopt{+} \hlstd{indicator2}
  \hlkwd{return}\hlstd{(Fx)}
\hlstd{\}}
\end{alltt}
\end{kframe}
\end{knitrout}
  %%%%%%%%%%%%%%%%%%%%%%%%%%%%%%%%%%%%%%%%%%%%%%%%%%%%%%%%%%%%%%%%%%%%%%%%%%%%%%%%%%%%%%%%%%%
  %%%%%%%%%  Part (b)
  %%%%%%%%%%%%%%%%%%%%%%%%%%%%%%%%%%%%%%%%%%%%%%%%%%%%%%%%%%%%%%%%%%%%%%%%%%%%%%%%%%%%%%%%%%%
	\item Show that you have a valid PMF. You can show this approximately by 
	calculating the series in a repeat loop until probability mass evaluations are 
	infinitesimally small.
	
	\textbf{Solution:} Simple mathematical validation that requires no coding. Will put it up here shortly
	%%%%%%%%%%%%%%%%%%%%%%%%%%%%%%%%%%%%%%%%%%%%%%%%%%%%%%%%%%%%%%%%%%%%%%%%%%%%%%%%%%%%%%%%%%%
  %%%%%%%%%  Part (c)
  %%%%%%%%%%%%%%%%%%%%%%%%%%%%%%%%%%%%%%%%%%%%%%%%%%%%%%%%%%%%%%%%%%%%%%%%%%%%%%%%%%%%%%%%%%%
	\item Find the median for your two sets of parameter(s). Conduct some research 
	to find the median based on our PMF to confirm that your numerical approach is
	correct. 
	\textbf{Solution:} Still a bit confused about this. Will put it up after office hours.
	%%%%%%%%%%%%%%%%%%%%%%%%%%%%%%%%%%%%%%%%%%%%%%%%%%%%%%%%%%%%%%%%%%%%%%%%%%%%%%%%%%%%%%%%%%%
  %%%%%%%%%  Part (d)
  %%%%%%%%%%%%%%%%%%%%%%%%%%%%%%%%%%%%%%%%%%%%%%%%%%%%%%%%%%%%%%%%%%%%%%%%%%%%%%%%%%%%%%%%%%%
	\item \label{q3PMF} Graph the PMF for several values of the parameter(s) 
	including the two sets you specified. What does changing the parameter(s) do 
	to the shape of the PMF?
	
	\textbf{Solutions:} We can plot the PMF of the Bernoulli distribution for different parameters.Note that the initially chosen parameters were $\mathrm{p =0.4}$ and $\mathrm{p=0.6}$. Let us also look at the PMF when the parameter values are 0.5, and 0.8.
	
\begin{knitrout}
\definecolor{shadecolor}{rgb}{0.969, 0.969, 0.969}\color{fgcolor}\begin{kframe}
\begin{alltt}
\hlkwd{library}\hlstd{(ggplot2)}
\hlkwd{library}\hlstd{(patchwork)}

\hlstd{plot.df} \hlkwb{<-} \hlkwd{data.frame}\hlstd{(}\hlkwc{x} \hlstd{= (}\hlopt{-}\hlnum{1}\hlopt{:}\hlnum{2}\hlstd{),}
                      \hlkwc{f1} \hlstd{=} \hlkwd{dbern}\hlstd{(}\hlkwc{x} \hlstd{= (}\hlopt{-}\hlnum{1}\hlopt{:}\hlnum{2}\hlstd{),} \hlkwc{prob} \hlstd{=} \hlnum{0.4}\hlstd{),}
                      \hlkwc{f2} \hlstd{=} \hlkwd{dbern}\hlstd{(}\hlkwc{x} \hlstd{= (}\hlopt{-}\hlnum{1}\hlopt{:}\hlnum{2}\hlstd{),} \hlkwc{prob} \hlstd{=} \hlnum{0.5}\hlstd{),}
                      \hlkwc{f3} \hlstd{=} \hlkwd{dbern}\hlstd{(}\hlkwc{x} \hlstd{= (}\hlopt{-}\hlnum{1}\hlopt{:}\hlnum{2}\hlstd{),} \hlkwc{prob} \hlstd{=} \hlnum{0.6}\hlstd{),}
                      \hlkwc{f4} \hlstd{=} \hlkwd{dbern}\hlstd{(}\hlkwc{x} \hlstd{= (}\hlopt{-}\hlnum{1}\hlopt{:}\hlnum{2}\hlstd{),} \hlkwc{prob} \hlstd{=} \hlnum{0.8}\hlstd{),}
                      \hlkwc{F1} \hlstd{=} \hlkwd{pbern}\hlstd{(}\hlkwc{q} \hlstd{= (}\hlopt{-}\hlnum{1}\hlopt{:}\hlnum{2}\hlstd{),} \hlkwc{prob} \hlstd{=} \hlnum{0.4}\hlstd{),}
                      \hlkwc{F2} \hlstd{=} \hlkwd{pbern}\hlstd{(}\hlkwc{q} \hlstd{= (}\hlopt{-}\hlnum{1}\hlopt{:}\hlnum{2}\hlstd{),} \hlkwc{prob} \hlstd{=} \hlnum{0.5}\hlstd{),}
                      \hlkwc{F3} \hlstd{=} \hlkwd{pbern}\hlstd{(}\hlkwc{q} \hlstd{= (}\hlopt{-}\hlnum{1}\hlopt{:}\hlnum{2}\hlstd{),} \hlkwc{prob} \hlstd{=} \hlnum{0.6}\hlstd{),}
                      \hlkwc{F4} \hlstd{=} \hlkwd{pbern}\hlstd{(}\hlkwc{q} \hlstd{= (}\hlopt{-}\hlnum{1}\hlopt{:}\hlnum{2}\hlstd{),} \hlkwc{prob} \hlstd{=} \hlnum{0.8}\hlstd{))}

\hlstd{PMF1} \hlkwb{<-} \hlkwd{ggplot}\hlstd{(}\hlkwc{data} \hlstd{= plot.df,} \hlkwd{aes}\hlstd{(}\hlkwc{x} \hlstd{= x))}\hlopt{+}
  \hlkwd{geom_linerange}\hlstd{(}\hlkwd{aes}\hlstd{(}\hlkwc{ymax} \hlstd{= f1),} \hlkwc{ymin} \hlstd{=} \hlnum{0}\hlstd{)}\hlopt{+}
  \hlkwd{geom_hline}\hlstd{(}\hlkwc{yintercept} \hlstd{=} \hlnum{0}\hlstd{)}\hlopt{+}
  \hlkwd{theme_bw}\hlstd{()} \hlopt{+}
  \hlkwd{xlab}\hlstd{(}\hlstr{"X"}\hlstd{)} \hlopt{+}
  \hlkwd{ylab}\hlstd{(}\hlkwd{bquote}\hlstd{(f[x](x)))} \hlopt{+}
  \hlkwd{ggtitle}\hlstd{(}\hlstr{"Bernoulli PMF"}\hlstd{,} \hlkwc{subtitle} \hlstd{=} \hlkwd{paste}\hlstd{(}\hlstr{"p ="}\hlstd{,} \hlnum{0.4}\hlstd{))}

\hlstd{PMF2} \hlkwb{<-} \hlkwd{ggplot}\hlstd{(}\hlkwc{data} \hlstd{= plot.df,} \hlkwd{aes}\hlstd{(}\hlkwc{x} \hlstd{= x))}\hlopt{+}
  \hlkwd{geom_linerange}\hlstd{(}\hlkwd{aes}\hlstd{(}\hlkwc{ymax} \hlstd{= f2),} \hlkwc{ymin} \hlstd{=} \hlnum{0}\hlstd{)}\hlopt{+}
  \hlkwd{geom_hline}\hlstd{(}\hlkwc{yintercept} \hlstd{=} \hlnum{0}\hlstd{)}\hlopt{+}
  \hlkwd{theme_bw}\hlstd{()} \hlopt{+}
  \hlkwd{xlab}\hlstd{(}\hlstr{"X"}\hlstd{)} \hlopt{+}
  \hlkwd{ylab}\hlstd{(}\hlkwd{bquote}\hlstd{(f[x](x)))} \hlopt{+}
  \hlkwd{ggtitle}\hlstd{(}\hlstr{"Bernoulli PMF"}\hlstd{,} \hlkwc{subtitle} \hlstd{=} \hlkwd{paste}\hlstd{(}\hlstr{"p ="}\hlstd{,} \hlnum{0.5}\hlstd{))}

\hlstd{PMF3} \hlkwb{<-} \hlkwd{ggplot}\hlstd{(}\hlkwc{data} \hlstd{= plot.df,} \hlkwd{aes}\hlstd{(}\hlkwc{x} \hlstd{= x))}\hlopt{+}
  \hlkwd{geom_linerange}\hlstd{(}\hlkwd{aes}\hlstd{(}\hlkwc{ymax} \hlstd{= f3),} \hlkwc{ymin} \hlstd{=} \hlnum{0}\hlstd{)}\hlopt{+}
  \hlkwd{geom_hline}\hlstd{(}\hlkwc{yintercept} \hlstd{=} \hlnum{0}\hlstd{)}\hlopt{+}
  \hlkwd{theme_bw}\hlstd{()} \hlopt{+}
  \hlkwd{xlab}\hlstd{(}\hlstr{"X"}\hlstd{)} \hlopt{+}
  \hlkwd{ylab}\hlstd{(}\hlkwd{bquote}\hlstd{(f[x](x)))} \hlopt{+}
  \hlkwd{ggtitle}\hlstd{(}\hlstr{"Bernoulli PMF"}\hlstd{,} \hlkwc{subtitle} \hlstd{=} \hlkwd{paste}\hlstd{(}\hlstr{"p ="}\hlstd{,} \hlnum{0.6}\hlstd{))}

\hlstd{PMF4} \hlkwb{<-} \hlkwd{ggplot}\hlstd{(}\hlkwc{data} \hlstd{= plot.df,} \hlkwd{aes}\hlstd{(}\hlkwc{x} \hlstd{= x))}\hlopt{+}
  \hlkwd{geom_linerange}\hlstd{(}\hlkwd{aes}\hlstd{(}\hlkwc{ymax} \hlstd{= f4),} \hlkwc{ymin} \hlstd{=} \hlnum{0}\hlstd{)}\hlopt{+}
  \hlkwd{geom_hline}\hlstd{(}\hlkwc{yintercept} \hlstd{=} \hlnum{0}\hlstd{)}\hlopt{+}
  \hlkwd{theme_bw}\hlstd{()} \hlopt{+}
  \hlkwd{xlab}\hlstd{(}\hlstr{"X"}\hlstd{)} \hlopt{+}
  \hlkwd{ylab}\hlstd{(}\hlkwd{bquote}\hlstd{(f[x](x)))} \hlopt{+}
  \hlkwd{ggtitle}\hlstd{(}\hlstr{"Bernoulli PMF"}\hlstd{,} \hlkwc{subtitle} \hlstd{=} \hlkwd{paste}\hlstd{(}\hlstr{"p ="}\hlstd{,} \hlnum{0.8}\hlstd{))}

\hlstd{PMF1} \hlopt{+} \hlstd{PMF2} \hlopt{+} \hlstd{PMF3} \hlopt{+} \hlstd{PMF4}
\end{alltt}
\end{kframe}
\end{knitrout}
\begin{figure}[H]
  \begin{center}
  % This code is evaluated, but not printed
  % note below I use message=FALSE and warning=FALSE to surpress what's printed
  % when running library(ggmap) or library(patchwork) which would otherwise cause
  % an error because Sweave is expecting just a graph (not a graph + text)
\begin{knitrout}
\definecolor{shadecolor}{rgb}{0.969, 0.969, 0.969}\color{fgcolor}
\includegraphics[width=\maxwidth]{figure/unnamed-chunk-2-1} 
\end{knitrout}
    \caption{The PMF of the Bernoulli distribution for different probability parameters p}
    \label{P3fig_1} %we can now reference P3fig_1
  \end{center}
\end{figure}

	%%%%%%%%%%%%%%%%%%%%%%%%%%%%%%%%%%%%%%%%%%%%%%%%%%%%%%%%%%%%%%%%%%%%%%%%%%%%%%%%%%%%%%%%%%%
  %%%%%%%%%  Part (e)
  %%%%%%%%%%%%%%%%%%%%%%%%%%%%%%%%%%%%%%%%%%%%%%%%%%%%%%%%%%%%%%%%%%%%%%%%%%%%%%%%%%%%%%%%%%%
	 \item Graph the CDF for the same values of the parameter(s) as you did in 
	 Question \ref{q3PMF}. What does changing the parameter(s) do to the shape of 
	 the CDF? Comment on the aspects of the CDFs that show that the CDF is valid.
	 
\begin{knitrout}
\definecolor{shadecolor}{rgb}{0.969, 0.969, 0.969}\color{fgcolor}\begin{kframe}
\begin{alltt}
\hlcom{# Handle open/closed points}
\hlstd{plot.df.openpoints} \hlkwb{<-} \hlkwd{data.frame}\hlstd{(}\hlkwc{x} \hlstd{= plot.df}\hlopt{$}\hlstd{x,}
                                 \hlkwc{y1} \hlstd{=} \hlkwd{pbern}\hlstd{(plot.df}\hlopt{$}\hlstd{x}\hlopt{-}\hlnum{1}\hlstd{,} \hlkwc{prob} \hlstd{=} \hlnum{0.4}\hlstd{),}
                                 \hlkwc{y2} \hlstd{=} \hlkwd{pbern}\hlstd{(plot.df}\hlopt{$}\hlstd{x}\hlopt{-}\hlnum{1}\hlstd{,} \hlkwc{prob} \hlstd{=} \hlnum{0.5}\hlstd{),}
                                 \hlkwc{y3} \hlstd{=} \hlkwd{pbern}\hlstd{(plot.df}\hlopt{$}\hlstd{x}\hlopt{-}\hlnum{1}\hlstd{,} \hlkwc{prob} \hlstd{=} \hlnum{0.6}\hlstd{),}
                                 \hlkwc{y4} \hlstd{=} \hlkwd{pbern}\hlstd{(plot.df}\hlopt{$}\hlstd{x}\hlopt{-}\hlnum{1}\hlstd{,} \hlkwc{prob} \hlstd{=} \hlnum{0.8}\hlstd{))}
\hlstd{plot.df.closedpoints} \hlkwb{<-} \hlkwd{data.frame}\hlstd{(}\hlkwc{x} \hlstd{= plot.df}\hlopt{$}\hlstd{x,}
                                   \hlkwc{y1} \hlstd{=} \hlkwd{pbern}\hlstd{(plot.df}\hlopt{$}\hlstd{x,}\hlkwc{prob}\hlstd{=}\hlnum{0.4}\hlstd{),}
                                   \hlkwc{y2} \hlstd{=} \hlkwd{pbern}\hlstd{(plot.df}\hlopt{$}\hlstd{x,}\hlkwc{prob}\hlstd{=}\hlnum{0.5}\hlstd{),}
                                   \hlkwc{y3} \hlstd{=} \hlkwd{pbern}\hlstd{(plot.df}\hlopt{$}\hlstd{x,}\hlkwc{prob}\hlstd{=}\hlnum{0.6}\hlstd{),}
                                   \hlkwc{y4} \hlstd{=} \hlkwd{pbern}\hlstd{(plot.df}\hlopt{$}\hlstd{x,}\hlkwc{prob}\hlstd{=}\hlnum{0.8}\hlstd{))}

\hlstd{CDF1} \hlkwb{<-} \hlkwd{ggplot}\hlstd{(}\hlkwc{data} \hlstd{= plot.df,} \hlkwd{aes}\hlstd{(}\hlkwc{x} \hlstd{= x,} \hlkwc{y} \hlstd{= F1))} \hlopt{+}
    \hlkwd{geom_step}\hlstd{()}\hlopt{+}
    \hlkwd{geom_point}\hlstd{(}\hlkwc{data} \hlstd{= plot.df.openpoints,} \hlkwd{aes}\hlstd{(}\hlkwc{x} \hlstd{= x,} \hlkwc{y} \hlstd{= y1),} \hlkwc{shape} \hlstd{=} \hlnum{1}\hlstd{)} \hlopt{+}
    \hlkwd{geom_point}\hlstd{(}\hlkwc{data} \hlstd{= plot.df.closedpoints,} \hlkwd{aes}\hlstd{(}\hlkwc{x} \hlstd{= x,} \hlkwc{y} \hlstd{= y1))} \hlopt{+}
    \hlkwd{theme_bw}\hlstd{()}\hlopt{+}
    \hlkwd{xlab}\hlstd{(}\hlstr{"X"}\hlstd{)}\hlopt{+}
    \hlkwd{ylab}\hlstd{(}\hlkwd{bquote}\hlstd{(F[x](x)))}\hlopt{+}
    \hlkwd{ggtitle}\hlstd{(}\hlstr{"Bernoulli CDF"}\hlstd{,}\hlkwc{subtitle}\hlstd{=(}\hlkwd{paste}\hlstd{(}\hlstr{"p ="}\hlstd{,} \hlnum{0.4}\hlstd{)))}

\hlstd{CDF2} \hlkwb{<-} \hlkwd{ggplot}\hlstd{(}\hlkwc{data} \hlstd{= plot.df,} \hlkwd{aes}\hlstd{(}\hlkwc{x} \hlstd{= x,} \hlkwc{y} \hlstd{= F2))} \hlopt{+}
    \hlkwd{geom_step}\hlstd{()}\hlopt{+}
    \hlkwd{geom_point}\hlstd{(}\hlkwc{data} \hlstd{= plot.df.openpoints,} \hlkwd{aes}\hlstd{(}\hlkwc{x} \hlstd{= x,} \hlkwc{y} \hlstd{= y2),} \hlkwc{shape} \hlstd{=} \hlnum{1}\hlstd{)} \hlopt{+}
    \hlkwd{geom_point}\hlstd{(}\hlkwc{data} \hlstd{= plot.df.closedpoints,} \hlkwd{aes}\hlstd{(}\hlkwc{x} \hlstd{= x,} \hlkwc{y} \hlstd{= y2))} \hlopt{+}
    \hlkwd{theme_bw}\hlstd{()}\hlopt{+}
    \hlkwd{xlab}\hlstd{(}\hlstr{"X"}\hlstd{)}\hlopt{+}
    \hlkwd{ylab}\hlstd{(}\hlkwd{bquote}\hlstd{(F[x](x)))}\hlopt{+}
    \hlkwd{ggtitle}\hlstd{(}\hlstr{"Bernoulli CDF"}\hlstd{,}\hlkwc{subtitle}\hlstd{=(}\hlkwd{paste}\hlstd{(}\hlstr{"p ="}\hlstd{,} \hlnum{0.5}\hlstd{)))}

\hlstd{CDF3} \hlkwb{<-} \hlkwd{ggplot}\hlstd{(}\hlkwc{data} \hlstd{= plot.df,} \hlkwd{aes}\hlstd{(}\hlkwc{x} \hlstd{= x,} \hlkwc{y} \hlstd{= F3))} \hlopt{+}
    \hlkwd{geom_step}\hlstd{()}\hlopt{+}
    \hlkwd{geom_point}\hlstd{(}\hlkwc{data} \hlstd{= plot.df.openpoints,} \hlkwd{aes}\hlstd{(}\hlkwc{x} \hlstd{= x,} \hlkwc{y} \hlstd{= y3),} \hlkwc{shape} \hlstd{=} \hlnum{1}\hlstd{)} \hlopt{+}
    \hlkwd{geom_point}\hlstd{(}\hlkwc{data} \hlstd{= plot.df.closedpoints,} \hlkwd{aes}\hlstd{(}\hlkwc{x} \hlstd{= x,} \hlkwc{y} \hlstd{= y3))} \hlopt{+}
    \hlkwd{theme_bw}\hlstd{()}\hlopt{+}
    \hlkwd{xlab}\hlstd{(}\hlstr{"X"}\hlstd{)}\hlopt{+}
    \hlkwd{ylab}\hlstd{(}\hlkwd{bquote}\hlstd{(F[x](x)))}\hlopt{+}
    \hlkwd{ggtitle}\hlstd{(}\hlstr{"Bernoulli CDF"}\hlstd{,}\hlkwc{subtitle}\hlstd{=(}\hlkwd{paste}\hlstd{(}\hlstr{"p ="}\hlstd{,} \hlnum{0.6}\hlstd{)))}

\hlstd{CDF4} \hlkwb{<-} \hlkwd{ggplot}\hlstd{(}\hlkwc{data} \hlstd{= plot.df,} \hlkwd{aes}\hlstd{(}\hlkwc{x} \hlstd{= x,} \hlkwc{y} \hlstd{= F4))} \hlopt{+}
    \hlkwd{geom_step}\hlstd{()}\hlopt{+}
    \hlkwd{geom_point}\hlstd{(}\hlkwc{data} \hlstd{= plot.df.openpoints,} \hlkwd{aes}\hlstd{(}\hlkwc{x} \hlstd{= x,} \hlkwc{y} \hlstd{= y4),} \hlkwc{shape} \hlstd{=} \hlnum{1}\hlstd{)} \hlopt{+}
    \hlkwd{geom_point}\hlstd{(}\hlkwc{data} \hlstd{= plot.df.closedpoints,} \hlkwd{aes}\hlstd{(}\hlkwc{x} \hlstd{= x,} \hlkwc{y} \hlstd{= y4))} \hlopt{+}
    \hlkwd{theme_bw}\hlstd{()}\hlopt{+}
    \hlkwd{xlab}\hlstd{(}\hlstr{"X"}\hlstd{)}\hlopt{+}
    \hlkwd{ylab}\hlstd{(}\hlkwd{bquote}\hlstd{(F[x](x)))}\hlopt{+}
    \hlkwd{ggtitle}\hlstd{(}\hlstr{"Bernoulli CDF"}\hlstd{,}\hlkwc{subtitle}\hlstd{=(}\hlkwd{paste}\hlstd{(}\hlstr{"p ="}\hlstd{,} \hlnum{0.8}\hlstd{)))}

\hlstd{CDF1} \hlopt{+} \hlstd{CDF2} \hlopt{+} \hlstd{CDF3} \hlopt{+} \hlstd{CDF4}
\end{alltt}
\end{kframe}
\end{knitrout}

\begin{figure}[H]
  \begin{center}
  % This code is evaluated, but not printed
  % note below I use message=FALSE and warning=FALSE to surpress what's printed
  % when running library(ggmap) or library(patchwork) which would otherwise cause
  % an error because Sweave is expecting just a graph (not a graph + text)
\begin{knitrout}
\definecolor{shadecolor}{rgb}{0.969, 0.969, 0.969}\color{fgcolor}
\includegraphics[width=\maxwidth]{figure/unnamed-chunk-3-1} 
\end{knitrout}
    \caption{The CDF of the Bernoulli distribution for different probability parameters p}
    \label{P3fig_2} %we can now reference P3fig_1
  \end{center}
\end{figure}
	%%%%%%%%%%%%%%%%%%%%%%%%%%%%%%%%%%%%%%%%%%%%%%%%%%%%%%%%%%%%%%%%%%%%%%%%%%%%%%%%%%%%%%%%%%%
  %%%%%%%%%  Part (f)
  %%%%%%%%%%%%%%%%%%%%%%%%%%%%%%%%%%%%%%%%%%%%%%%%%%%%%%%%%%%%%%%%%%%%%%%%%%%%%%%%%%%%%%%%%%%
  \item Generate a random sample of size $n=10, 25, 100$, and $1000$ for your 
  two sets of parameter(s). In a $4 \times 2$ grid, plot a histogram (with bin 
  size 1) of each set of data and superimpose the true mass function at the 
  specified parameter values. Interpret the results.
	\end{enumerate}
%%%%%%%%%%%%%%%%%%%%%%%%%%%%%%%%%%%%%%%%%%%%%%%%%%%%%%%%%%%%%%%%%%%%%%%%%%%%%%%%%%%%%%%%%%%
%%%%%%%%%%%%%%%%%%%%%%%%%%%%%%%%%%%%%%%%%%%%%%%%%%%%%%%%%%%%%%%%%%%%%%%%%%%%%%%%%%%%%%%%%%%
%%%%%%%%%  Question 2
%%%%%%%%%%%%%%%%%%%%%%%%%%%%%%%%%%%%%%%%%%%%%%%%%%%%%%%%%%%%%%%%%%%%%%%%%%%%%%%%%%%%%%%%%%%
%%%%%%%%%%%%%%%%%%%%%%%%%%%%%%%%%%%%%%%%%%%%%%%%%%%%%%%%%%%%%%%%%%%%%%%%%%%%%%%%%%%%%%%%%%%
\item Continue with the discrete distribution you selected for Question \ref{Q3}.
\begin{enumerate}
  %%%%%%%%%%%%%%%%%%%%%%%%%%%%%%%%%%%%%%%%%%%%%%%%%%%%%%%%%%%%%%%%%%%%%%%%%%%%%%%%%%%%%%%%%%%
  %%%%%%%%%  Part (a)
  %%%%%%%%%%%%%%%%%%%%%%%%%%%%%%%%%%%%%%%%%%%%%%%%%%%%%%%%%%%%%%%%%%%%%%%%%%%%%%%%%%%%%%%%%%%
  \item Provide the mean, standard deviation, skewness, and kurtosis of the PMF. 
  Ensure to interpret each.
  %%%%%%%%%%%%%%%%%%%%%%%%%%%%%%%%%%%%%%%%%%%%%%%%%%%%%%%%%%%%%%%%%%%%%%%%%%%%%%%%%%%%%%%%%%%
  %%%%%%%%%  Part (b)
  %%%%%%%%%%%%%%%%%%%%%%%%%%%%%%%%%%%%%%%%%%%%%%%%%%%%%%%%%%%%%%%%%%%%%%%%%%%%%%%%%%%%%%%%%%%
  \item Generate a random sample of size $n=10, 25, 100$, and $1000$ for your 
  two sets of parameter(s). Calculate the sample mean, standard deviation, 
  skewness, and kurtosis. Interpret the results.
  %%%%%%%%%%%%%%%%%%%%%%%%%%%%%%%%%%%%%%%%%%%%%%%%%%%%%%%%%%%%%%%%%%%%%%%%%%%%%%%%%%%%%%%%%%%
  %%%%%%%%%  Part (c)
  %%%%%%%%%%%%%%%%%%%%%%%%%%%%%%%%%%%%%%%%%%%%%%%%%%%%%%%%%%%%%%%%%%%%%%%%%%%%%%%%%%%%%%%%%%%
  \item Generate a random sample of size $n=10$ for your two sets of parameter(s).
  Calculate the method of moments estimator(s) and maximum likelihood estimator(s).
  In a $1 \times 2$ grid, plot a histogram (with bin size 1) of each set of data 
  with (1) the method of moments estimated distribution, (2) the maximum likelihood 
  estimated distribution, and superimpose the true distribution in both.
  %%%%%%%%%%%%%%%%%%%%%%%%%%%%%%%%%%%%%%%%%%%%%%%%%%%%%%%%%%%%%%%%%%%%%%%%%%%%%%%%%%%%%%%%%%%
  %%%%%%%%%  Part (d)
  %%%%%%%%%%%%%%%%%%%%%%%%%%%%%%%%%%%%%%%%%%%%%%%%%%%%%%%%%%%%%%%%%%%%%%%%%%%%%%%%%%%%%%%%%%%
  \item Generate a random sample of size $n=25$ for your two sets of parameter(s). 
  Calculate the method of moments estimator(s) and maximum likelihood estimator(s).
  In a $1 \times 2$ grid, plot a histogram (with bin size 1) of each set of data 
  with (1) the method of moments estimated distribution, (2) the maximum likelihood 
  estimated distribution, and superimpose the true distribution in both.
  %%%%%%%%%%%%%%%%%%%%%%%%%%%%%%%%%%%%%%%%%%%%%%%%%%%%%%%%%%%%%%%%%%%%%%%%%%%%%%%%%%%%%%%%%%%
  %%%%%%%%%  Part (e)
  %%%%%%%%%%%%%%%%%%%%%%%%%%%%%%%%%%%%%%%%%%%%%%%%%%%%%%%%%%%%%%%%%%%%%%%%%%%%%%%%%%%%%%%%%%%
  \item Generate a random sample of size $n=100$ for your two sets of parameter(s).
  Calculate the method of moments estimator(s) and maximum likelihood estimator(s). 
  In a $1 \times 2$ grid, plot a histogram (with bin size 1) of each set of data 
  with (1) the method of moments estimated distribution, (2) the maximum likelihood
  estimated distribution, and superimpose the true distribution in both.
  %%%%%%%%%%%%%%%%%%%%%%%%%%%%%%%%%%%%%%%%%%%%%%%%%%%%%%%%%%%%%%%%%%%%%%%%%%%%%%%%%%%%%%%%%%%
  %%%%%%%%%  Part (f)
  %%%%%%%%%%%%%%%%%%%%%%%%%%%%%%%%%%%%%%%%%%%%%%%%%%%%%%%%%%%%%%%%%%%%%%%%%%%%%%%%%%%%%%%%%%%
  \item Generate a random sample of size $n=100$ for your two sets of parameter(s).
  Calculate the method of moments estimator(s) and maximum likelihood estimator(s).
  In a $1 \times 2$ grid, plot a histogram (with bin size 1) of each set of data 
  with (1) the method of moments estimated distribution, (2) the maximum likelihood
  estimated distribution, and superimpose the true distribution in both.
  %%%%%%%%%%%%%%%%%%%%%%%%%%%%%%%%%%%%%%%%%%%%%%%%%%%%%%%%%%%%%%%%%%%%%%%%%%%%%%%%%%%%%%%%%%%
  %%%%%%%%%  Part (g)
  %%%%%%%%%%%%%%%%%%%%%%%%%%%%%%%%%%%%%%%%%%%%%%%%%%%%%%%%%%%%%%%%%%%%%%%%%%%%%%%%%%%%%%%%%%%
  \item Comment on the results of parts (c)-(f). 
\end{enumerate}
\end{enumerate}%End overall enumerate
\newpage
\bibliography{bib}
\end{document}
